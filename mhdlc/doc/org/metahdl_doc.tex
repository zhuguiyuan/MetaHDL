% Intended LaTeX compiler: pdflatex
\documentclass[10pt,oneside]{book}
\usepackage{MXBookphv}
\makeindex
\usepackage[utf8]{inputenc}
\usepackage[T1]{fontenc}
\usepackage{graphicx}
\usepackage{longtable}
\usepackage{wrapfig}
\usepackage{rotating}
\usepackage[normalem]{ulem}
\usepackage{amsmath}
\usepackage{amssymb}
\usepackage{capt-of}
\usepackage{hyperref}
\author{xinmeng@hotmail.com}
\date{2023-11-27}
\title{MetaHDL Manual}
\hypersetup{
 pdfauthor={xinmeng@hotmail.com},
 pdftitle={MetaHDL Manual},
 pdfkeywords={},
 pdfsubject={},
 pdfcreator={Emacs 29.4 (Org mode 9.6.15)}, 
 pdflang={English}}
\begin{document}

\maketitle
\tableofcontents


\chapter{Introduction}
\label{sec:org00340b3}
MetaHDL (shorted as "mhdl" in this document) is an HDL aims at
synthesizable digital VLSI designs (commonly known as RTL designs).
mhdl selectively inherits SystemVerilog
syntax, eliminates unnecessary variants, extends existing
synthesizable language structures and adds new grammars to simplify
RTL coding.  Designers will find it quite intuitive and flexible when
using mhdl.  A compiler named \texttt{mhdlc} is implemented to translate mhdl
to SystemVerilog or Verilog

\section{Features}
\label{sec:orgc968f7e}
\begin{enumerate}
\item Comprehensive Preprocessor

\item Flexible declarations

\item Port inference and automatic variable declarations

\item Enhanced instantiation syntax

\item New syntax for ff and fsm

\item Parameter tracing

\item Automatic dependency resolving

\item Lightweight lint checking

\item Independent Verilog Parser to support IP integration

\item Rich user control syntax

\item Re-indent the generated sv/v
\end{enumerate}

\section{Document Organization}
\label{sec:org71a7d4f}
In the reset of this manual, mhdl syntax and usage will be documented in
detail, following is the organization of this document:

\begin{itemize}
\item \hyperref[sec:org0a37dbd]{Basic Concepts} gives many basic and important concepts behind
mhdl.  All readers are expected to read this chapter carefully,
otherwise, later chapters are difficult to understand.

\item \hyperref[sec:org7a8e6f7]{Syntax} gives major syntax explanations and sample codes. After
reading this chapter, readers can develop complex chips with
powerful capabilities provided by mhdl.

\item \hyperref[sec:orga49f1d8]{Preprocessor} describes preprocessor in mhdlc and support
directives. Designers can achieve script-like code configurations by
using this build-in preprocessor, instead of writing dozens of
one-time scripts.

\item \hyperref[sec:org17bead1]{User Control} lists various user control
variables that alter compiler execution.

\item \hyperref[sec:orgcd94bb9]{Command Line Options} documents all command line options accepted by
mhdlc.

\item \hyperref[sec:orgd4e6ee2]{For VPerl Designers} provides additional information for those who
originally use vperl for daily coding.
Differences with vperl are summarized there.

\item \hyperref[sec:orgbcbe461]{formal syntax} is the complete formal syntax of mhdl.
\end{itemize}

\section{download and bug report}
\label{sec:org525a051}
\texttt{mhdlc} is publically available at \url{https://github.com/xinmeng/metahdl}.
if you find any bug of mhdlc, ambiguous contents or typo in this
document, please file issue at github. 

\chapter{Basic Concepts}
\label{sec:org0a37dbd}
RTL designs are not like other programming, there are few local
variables.  In addition to physical resources occupation semantics,
RTL variables also represent nets or connections.  Physical elements
are connected via variables.  Normally, there is no floating net
inside modules, which means every net should have source and sinks.
if a net has no source, it should be module input port, which will be
fed by external drivers.  If it has no sink, it should be module
output port, which will drive external modules' signal.  If it has
both source and sinks, it is most probably an internal net.  These are
basic rules of port inference in mhdl.  Designers can override these
rules by adding explicit port declarations.

Module ports are automatically inferred by compiler, and designers can
ask compiler to perform port validation (or port checking) against
designers' explicit declaration.  In this scenario, golden ports are
declared by designers, compiler compares the inferred ports and
declared ports, any missing or newly emerging ports are reported as
error.

In mhdl world, there are \textbf{only} four types of building blocks in
synthesizable rtl designs:

\begin{enumerate}
\item combinational logic
\item sequential logic (mostly flip-flop)
\item module instantiation
\item FSM.  Technically, FSM is essentially a mixture of combinational
and sequential logic, because it is so commonly used, we promote it
to a basic structure.
\end{enumerate}

They are called \emph{code block} in the rest of this document.  Any
modules, no matter how complex it is, can be decomposed to these four
structures.  Module is treated as a physical resources wrapper with
parameters to be overridden upon instantiation.

mhdl RTL designing is a process in which designers describe
functionalities using code blocks.  \texttt{mhdlc} connects nets with same
name and infers ports according to designers' declarations.
Parameters are recognized and ports/nets are parameterized
automatically.

mhdl also allows designers to embed script-like code configurations
(via preprocessor) in RTL in a reuse oriented design.  Module logic
can be fine-grain tuned before translating mhdl to SystemVerilog.
Ports and variable declarations are dynamically updated according to
logic configuration.

\chapter{Syntax}
\label{sec:org7a8e6f7}
mhdl selectively inherits synthesizable syntax of SystemVerilog,
eliminates unnecessary variants, extends module instantiation syntax,
add new syntax for flip-flop and FSM.  Verilog or SystemVerilog
designers will find it quite intuitive to use mhdl syntax.  In the rest
of this chapter, major syntax are presented with examples, refer to
\hyperref[sec:orgbcbe461]{Formal Syntax} for complete syntax.

\section{Combinational Logic}
\label{sec:org5bdcb1d}
There are two and \textbf{only two} types of code block in mhdl for coding
combinational logic, as shown in \autoref{code:comb}:

\begin{enumerate}
\item \texttt{assign} statement
\item \texttt{always\_comb} statement
\end{enumerate}

\begin{lstlisting}[language=verilog,label= ,caption={Combinational logic example},captionpos=b,firstnumber=1,numbers=left,label={code:comb},frame=single]
// OK, accepted
always_comb
  if ( enabled )
    o1 = i1 | i2 | i3;
  else
    o1 = 1'b0;

// OK, accepted
assign o2 = cond ? i1 : i2;

// Illegal, wrong!!
// conventional Verilog is NOT supported
always @( i1 or i2 or i3 )
  if ( enabled )
    o1 = i1 | i2 | i3;
  else
    o1 = 1'b0;


// Illegal, wrong!!
// Verilog 2000 is NOT supported, either
always @(*)
  if ( enabled )
    o1 = i1 | i2 | i3;
  else
    o1 = 1'b0;  
\end{lstlisting}

\section{Sequential Logic}
\label{sec:org1aa85b8}
There are two types of code block in mhdl for coding sequential logic:

\begin{enumerate}
\item \texttt{always\_ff @()} statement, which is same in SystemVerilog
\item \texttt{ff-endff} block, which is introduced by mhdl
\end{enumerate}

The troditional SystemVerilog syntax is good except its redundancy: FF
variable appears twice (even more) in different clause of \texttt{if-else}
branches.  For a multi-bit vector variable, such redundancy is prone
to typo and width mismatch.

For well-coded FF, combinational part of the FF sources should be
coded in a separate code block, so the \texttt{if-else} branches can be
reduced.  mhdl provides a new \texttt{ff-endff} code block to reduce
redundancy.  FF code in following two forms (\autoref{code:ff_blk}) are
equivalent.

\begin{lstlisting}[language=verilog,label= ,caption={ff-endff code block example},captionpos=b,firstnumber=1,numbers=left,label={code:ff_blk}]
// troditional sequential block
always_ff @ (posedge clk or negedge rst_n) 
  if (!rst_n) 
    a_ff <= 1'b0; 
  else
    a_ff <= a;

// MetaHDL new sequential block
ff; 
  a_ff, a, 1'b0; 
endff
\end{lstlisting}


\texttt{ff-endff} block can optionally specify clock and reset signal name.
Usually they are ommitted to further reduce redundancy.  Each
line in block describes a FF.  A line has three element:

\begin{enumerate}
\item FF variable name, \texttt{a\_ff} here.
\item An expression containing the logic to update the FF, \texttt{a} here.  Any
expression defined in \hyperref[sec:orgbcbe461]{Formal Syntax} are allowed here.
\item An optinally reset value.  If no reset value is provided, FF
variable will not be reset.
\end{enumerate}

Here is example:
\begin{lstlisting}[language=verilog,label= ,caption={ff-endff block example and corresponding SystemVerilog},captionpos=b,firstnumber=1,numbers=left]
ff clk_a, clk_a_rst_n;
  a_ff, a, 1'b0;
  b_ff, b;
  c_ff, a_ff & b_ff, 1'b0;
endff

// Generated SystemVerilog from above code
always_ff @(posedge clk_a or negedge clk_a_rst_n)
  if (!clk_a_rst_n) begin
     a_ff <= 1'b0; 
     c_ff <= 1'b0; 
  end 
  else begin
     a_ff <= a; 
     b_ff <= b; 
     c_ff <= a_ff & b_ff; 
  end
\end{lstlisting}


\section{FSM}
\label{sec:orgac0159b}
FSM in conventional RTL design requires many constant/parameter
definitions to improve code readability.  But these definitions are
tedious to maintain during develop iterations, especially for one-hot
encoded FSM.  mhdl introduces \emph{symbol based} FSM programming paradigm
that liberates designers from such frustrated situation.

FSM code block is enclosed by keywords \texttt{fsm}, \texttt{fsm\_nc} and \texttt{endfsm}.
\texttt{fsm} keyword is followed by three identifiers:

\begin{enumerate}
\item FSM name, which is mandatory.
\item clock signal name, which is optional.
\item reset signal name, which is optional, too.
\end{enumerate}

FSM name is used as based name of state register, \texttt{\_cs} and \texttt{\_ns}
suffix are appended to FSM name to create current state register and
next state next state register, respectively.  Clock and reset signal
names are used in sequential block of FSM, which resets state register
and perform current state refreshing.  Clock and reset names can be
omitted together, and default name \texttt{clk} and \texttt{rst\_n} will be used.
State transition is explicitly stated by \texttt{goto} keyword, instead of
next state assignment.

Symbol based FSM programming allows designers to code FSM using state
names, one-hot state encodings are automatically generated by \texttt{mhdlc}.
Constant definitions are generated according to state names to improve
code readability.  To help designers eliminate state name typo, mhdlc
will build a \emph{Directed Graph} representing state transition during
parsing.  By checking the connectivity of every state, dead states and
unreachable states are reported to designers for confirmation.
\autoref{code:fsm_mhdl} is mhdl FSM description, \autoref{code:fsm_sv} is
the corresponding SystemVerilog description, including constant
definition.

\begin{lstlisting}[language=verilog,label= ,caption={FSM code in MetaHDL},captionpos=b,firstnumber=1,numbers=left]
fsm cmdrx, clk, rst_n; 

cm_pim_ack = 1'b0;

IDLE: begin
   if ( pim_cm_req ) begin
      cm_pim_ack = 1'b1; 
      goto DATA; 
   end
   else begin
      goto IDLE; 
   end
end

DATA: begin
   cm_pim_ack = 1'b1; 
   if ( pim_cm_eof ) begin
      cm_pim_ack = 1'b0; 
      goto IDLE; 
   end
   else
     begin
        goto DATA; 
     end
end

endfsm
\end{lstlisting}

\begin{lstlisting}[language=verilog,label= ,caption={FSM code generated in SystemVerilog},captionpos=b,firstnumber=1,numbers=left]
// other declarations...
const logic [1:0] DATA = 2'b10; 
const logic [1:0] IDLE = 2'b01; 
const int _DATA_ = 1; 
const int _IDLE_ = 0;

// Sequential part of cmdrx
always_ff @(posedge clk or negedge rst_n) 
  if( ~rst_n) begin
     cmdrx_cs <= IDLE; 
  end 
  else begin
     cmdrx_cs <= cmdrx_ns;
  end

// Combnational part of cmdrx 
always_comb begin 
   cm_pim_ack = 1'b0;
   unique case ( 1'b1 ) 
     cmdrx_cs[_IDLE_] : begin 
        if ( pim_cm_req ) begin
           cm_pim_ack = 1'b1; 
           cmdrx_ns = DATA; 
        end
        else begin
           cmdrx_ns = IDLE; 
        end
     end

     cmdrx_cs[_DATA_] : begin
        cm_pim_ack = 1'b1; 
        if ( pim_cm_eof ) begin
           cm_pim_ack = 1'b0; 
           cmdrx_ns = IDLE; 
        end
        else begin
           cmdrx_ns = DATA; 
        end
     end

     default: begin 
        cmdrx_ns = 2'hX; 
     end 
   endcase
end
\end{lstlisting}

Difference between \texttt{fsm} and \texttt{fsm\_nc} is Verilog generated from
\texttt{fsm\_nc} block will not contain the sequential block.  That means
designers have to manually code the sequential block.  This is
expecially designed for FSM with synchronous reset.  \textbf{Note} that the
manual crafted sequential block \emph{must} come after the block, because
and signals are only accessible after block is parsed.

\section{Module Instantiation}
\label{sec:org2674b28}
SystemVerilog module instantiation syntax is extended in mhdl, BNF is
shown in \hyperref[sec:orgbcbe461]{Formal Syntax}, start from non-terminal "inst\_block". Features
of mhdl instantiation syntax are highlighted below:

\begin{enumerate}
\item Instance name is optional.  Default instance name created by
prefixeding \texttt{x\_} on module named.
\item Port connection is optional.  Default behavior is to connect ports to
net with identical name.
\item Prefix and/or Suffix connection rules are allowed in port
connection (see example below).
\item Regular expression connection rule is allowed in port connection
(see example below).
\end{enumerate}

\begin{lstlisting}[language=verilog,label= ,caption={Module \texttt{moda} to be instantiated},captionpos=b,firstnumber=1,numbers=left,label={code:moda}]
module moda
  (input i1,
   input i2, 
   output o1, 
   output [1:0] o2
   );

endmodule
\end{lstlisting}

\autoref{code:inst_moda_mhdl} shows different connection rules.
\autoref{code:inst_moda_v} is the generated SystemVerilog. 

\begin{lstlisting}[language=verilog,label= ,caption={Instantiate moda in mhdl},captionpos=b,firstnumber=1,numbers=left,label={code:inst_moda_mhdl}]
// simplest instantiation
moda;

// prefix connection rule
moda x1_moda ( x1_ +);

// suffix connection rule after prefix rule
moda x2_moda ( x2_ + , 
               + _22);

// regexp connection rule
moda x3_moda
  ( "s/o/out/g", 
    "s/i/in/g" );
\end{lstlisting}


\begin{lstlisting}[language=verilog,label= ,caption={Instaniate \texttt{moda}},captionpos=b,firstnumber=1,numbers=left,label={code:inst_moda_v}]
moda x_moda
  (.i1 (i1),
   .i2 (i2),
   .o1 (o1),
   .o2 (o2) );

moda x1_moda
  ( .i1 (x1_i1),
    .i2 (x1_i2),
    .o1 (x1_o1),
    .o2 (x1_o2) );

moda x2_moda
  ( .i1 (x2_i1_22),
    .i2 (x2_i2_22),
    .o1 (x2_o1_22),
    .o2 (x2_o2_22) );

moda x3_moda
  ( .i1 (in1),
    .i2 (in2),
    .o1 (out1),
    .o2 (out2) );
\end{lstlisting}

\section{Parameter Tracing}
\label{sec:org972f3d1}
mhdl enables designers to creates parameterized module in two ways:
\begin{enumerate}
\item Write parameterized module in MetaHDl from draft.
\item Build parameterized module by instantiating parameterized modules.
\end{enumerate}

To use parameter in mhdl source code, designers declare parameters,
and use them in ports or net.  \texttt{mhdlc} will automatically parameterize
ports in generated declarations.  If a module to be instantiated is a
parameterized module, \texttt{mhdlc} can trace parameter usage in port
connections and automatically parameterize nets in wrapper
module. Here are some examples. 


In \autoref{code:modc_mhdl}, \texttt{modc} is created in mhdl with parameters.
\autoref{code:modc_v} is the generated SystemVerilog.  Ports and nets
are properly parameterized. 

\begin{lstlisting}[language=verilog,label= ,caption={Coding \texttt{modc} in mhdl with parameters},captionpos=b,firstnumber=1,numbers=left,label={code:modc_mhdl}]
parameter A = 4;
parameter B = 5;
parameter C = A + B;

assign o1[C-1:0] = {~i1[A-1:0], i2[B-1:0]};
\end{lstlisting}

\begin{lstlisting}[language=verilog,label= ,caption={Generated modc.v with parameters},captionpos=b,firstnumber=1,numbers=left,label={code:modc_v}]
module modc
  ( i1, i2, o1);

   parameter A = 4; 
   parameter B = 5; 
   parameter C = 4 + 5;

   input [A - 1:0] i1; 
   input [B - 1:0] i2; 
   output [C - 1:0] o1;

   logic [A - 1:0]  i1; 
   logic [B - 1:0]  i2; 
   logic [C - 1:0]  o1;

   assign o1[C - 1:0] = {~i1[A - 1:0], i2[B - 1:0]};

endmodule
\end{lstlisting}


In \autoref{code:inst_modc_mhdl}, \texttt{modc} is instantiated several times
to demonstrate parameter tracing. 

\begin{lstlisting}[language=verilog,label= ,caption={Instantiate \texttt{modc} in mhdl with parameter override},captionpos=b,firstnumber=1,numbers=left,label={code:inst_modc_mhdl}]
parameter SETA = 8;
parameter SETB = 9;

modc #(.A(2)) x0_modc ( x0_ + );

modc #(.A (SETA), .B (SETB)) x1_modc ( x1_ + );

modc #(.A (SETA)) x2_modc (x2_ +, .o1 (x2_o1[10:0]));
\end{lstlisting}

\begin{lstlisting}[language=verilog,label= ,caption={Genrated wrapper with parameter tracing},captionpos=b,firstnumber=1,numbers=left,label={code:inst_modc_v}]
module modwrapper
  ( x0_i1, x0_i2, x0_o1, x1_i1, x1_i2, x1_o1, x2_i1, x2_i2, x2_o1);

parameter SETA = 8; 
parameter SETB = 9;

   input  [1 :0]                x0_i1;
   input [4 :0]                 x0_i2;
   output [6 :0]                x0_o1;
   input [SETA - 1:0]           x1_i1;
   input [SETB - 1:0]           x1_i2;
   output [SETA + SETB - 1:0]   x1_o1;
   input [SETA - 1:0]           x2_i1;
   input [4 :0]                 x2_i2;
   output [10 :0]               x2_o1;

   logic [1 :0]                 x0_i1;
   logic [4 :0]                 x0_i2;
   logic [6 :0]                 x0_o1;
   logic [SETA - 1:0]           x1_i1;
   logic [SETB - 1:0]           x1_i2;
   logic [SETA + SETB - 1:0]    x1_o1;
   logic [SETA - 1:0]           x2_i1;
   logic [4 :0]                 x2_i2;
   logic [10 :0]                x2_o1;

   modc #( .A( 2 ), 
           .B( 5 ), 
           .C( 2 + 5 ) ) 
   x0_modc
     ( .i1 (x0_i1),
       .i2 (x0_i2), 
       .o1 (x0_o1) );
   
   modc #( .A( SETA ), 
           .B( SETB ), 
           .C( SETA + SETB ) ) 
   x1_modc
     ( .i1 (x1_i1), 
       .i2 (x1_i2),
       .o1 (x1_o1) );

   modc #( .A( SETA ), 
           .B( 5 ), 
           .C( SETA + 5 ) ) 
   x2_modc
     ( .i1 (x2_i1), 
       .i2 (x2_i2),
       .o1 (x2_o1[10:0]) );

endmodule
\end{lstlisting}

\section{Optional Declaration}
\label{sec:org4348cb9}
Declaration in Verilog and SystemVerilog is mandatory, but in mhdl is
optional.  \texttt{mhdlc} can automatcially infer width, port directions, and
variable type from a well designed synthesizable RTL code.  But in
some cases, designers want to constrain the inference results.  This can
be done by declaration statements.  Usually, declaration is used in
follow sceanrios:

\begin{enumerate}
\item Force port direction, such as
\begin{lstlisting}[language=verilog,label= ,caption= ,captionpos=b,numbers=none]
input a;
output b;
nonport c;
\end{lstlisting}

\item 2 dimensional array, such as
\begin{lstlisting}[language=verilog,label= ,caption= ,captionpos=b,numbers=none]
reg [31:0] unpacked_a [15:0];
reg [31:0] [7:0] packed_b;

always_comb begin
   // i will be declared by mhdlc
   for (i=0; i<8; i++)
     packed_b[i] = unpacked_a[i];
end
\end{lstlisting}
\end{enumerate}

\chapter{Preprocessor}
\label{sec:orga49f1d8}
Preprocessor helps designers to embed script-like code configuration
directives into RTL code for reuse oriented designs.  Conventionally,
designers are used to write one-time scripts (Perl/sed/awk/csh) to
preprocess their RTL for using in different project.  This methodology
is not clean enough.  Verification engineers have to create additional
steps in Makefile to preprocess code.  mhdl preprocessor uses
SystemVerilog style macro syntax, introduces more flow control
directives that help designers perform conditional and repetitive
configuration on RTL.

In addition to conventional \texttt{`ifdef}, \texttt{`ifndef}, \texttt{`else}, \texttt{`define}
and \texttt{`include} macro directives, introduces \texttt{`for}, \texttt{`if} and \texttt{`let}
to enlarge the power of preprocessor (see following examples).

\autoref{code:pp_arb} is a simple Round Robin Arbiter FSM implemented in
with facilitating preprocessor.  This arbiter can respond to a
configurable number of slaves, which is controlled by macro
\texttt{SLV\_NUM}.  Once the code is finished, various arbiters can be
generated by giving different values to \texttt{SLV\_NUM}.

\begin{lstlisting}[language=verilog,label= ,caption={Configurable Arbiter in mhdl},captionpos=b,firstnumber=1,numbers=left,label={code:pp_arb}]
fsm arb;

`for (i=1; `i<=`SLV NUM; i++)
slave_grnt_`i = 1'b0;
`endfor

`for (i=1; `i<=`SLV NUM; i++)
 `let j = `i + 1

 `if `i != `SLV NUM
SLAVE_`i: begin
   if ( slave_req_`i ) begin
      slave_grnt_`i = 1'b1;
      if ( slave_eof_`i ) begin
         slave_grnt_`i = 1'b0;
         goto SLAVE_`j;
      end
      else begin
         goto SLAVE_`i;
      end
   end
   else
     goto SLAVE_`j;
end

 `else
SLAVE_`i: begin
   if ( slave_req_`i ) begin
      slave_grnt_`i = 1'b1;
      if ( slave_eof_`i ) begin
         slave_grnt_`i = 1'b0;
         goto SLAVE_1;
      end
      else
        goto SLAVE_`i;
   end
   else
     goto SLAVE_1;
end
 `endif
`endfor
endfsm
\end{lstlisting}

A \texttt{`for} directive is used to repetitively "write" code with slight
difference.  Default values of FSM output are set within this block.

Another \texttt{`for} directive to write slave handling code, one state for
each slave.  Since it is a Round Robin arbiter, every states perform
same task: grant slave access if has request, move to next slave when
current one has no request or transaction is done, roll back to the
first slave when a arbitration round finishes.  A \texttt{`if} block is used
to check whether current state is for last slave.

A \texttt{`let} directive is used to perform arithmetic operation and
calculate value of \texttt{`j}, which is the number of next slave.

\autoref{code:pp_arb_v} shows te SyntemVerilog generated with \texttt{SLV\_NUM=4}.

\begin{lstlisting}[language=verilog,label= ,caption={Generated Arbiter in SystemVerilog code (SLV\_NUM=4)},captionpos=b,firstnumber=1,numbers=left,label={code:pp_arb_v}]
module arbiter (
                clock,
                reset_n,
                slave_eof_1,
                slave_eof_2,
                slave_eof_3,
                slave_eof_4,
                slave_grnt_1,
                slave_grnt_2,
                slave_grnt_3,
                slave_grnt_4,
                slave_req_1,
                slave_req_2,
                slave_req_3,
                slave_req_4);


   input clock;
   input reset_n;
   input slave_eof_1;
   input slave_eof_2;
   input slave_eof_3;
   input slave_eof_4;
   output slave_grnt_1;
   output slave_grnt_2;
   output slave_grnt_3;
   output slave_grnt_4;
   input  slave_req_1;
   input  slave_req_2;
   input  slave_req_3;
   input  slave_req_4;

   const logic [3:0] SLAVE_1 = 4'b0001;
   const logic [3:0] SLAVE_2 = 4'b0010;
   const logic [3:0] SLAVE_3 = 4'b0100;
   const logic [3:0] SLAVE_4 = 4'b1000;
   const int         _SLAVE_1_ = 0;
   const int         _SLAVE_2_ = 1;
   const int         _SLAVE_3_ = 2;
   const int         _SLAVE_4_ = 3;
   logic [3:0]       arb_cs;
   logic [3:0]       arb_ns;
   logic             clock;
   logic             reset_n;
   logic             slave_eof_1;
   logic             slave_eof_2;
   logic             slave_eof_3;
   logic             slave_eof_4;
   logic             slave_grnt_1;
   logic             slave_grnt_2;
   logic             slave_grnt_3;
   logic             slave_grnt_4;
   logic             slave_req_1;
   logic             slave_req_2;
   logic             slave_req_3;
   logic             slave_req_4;

   // Sequential part of FSM /tmp/xin_meng/mhdlc/test/arbiter.mhdl:1.0-42.5
   // /tmp/xin_meng/mhdlc/test/arbiter.mhdl:1.0-42.5
   always_ff @(posedge clock or negedge reset_n)
     if (~reset_n) begin
        arb_cs <= SLAVE_1;
     end
     else begin
        arb_cs <= arb_ns;
     end

   // Combnational part of FSM /tmp/xin_meng/mhdlc/test/arbiter.mhdl:1.0-42.5
   // /tmp/xin_meng/mhdlc/test/arbiter.mhdl:1.0-42.5
   always_comb begin
      slave_grnt_1 = 1'b0;
      slave_grnt_2 = 1'b0;
      slave_grnt_3 = 1'b0;
      slave_grnt_4 = 1'b0;
      unique case ( 1'b1 )
        arb_cs[_SLAVE_1_] : begin
           if ( slave_req_1 ) begin
              slave_grnt_1 = 1'b1;
              if ( slave_eof_1 ) begin
                 slave_grnt_1 = 1'b0;
                 arb_ns = SLAVE_2;
              end
              else begin
                 arb_ns = SLAVE_1;
              end
           end
           else begin
              arb_ns = SLAVE_2;
           end
        end

        arb_cs[_SLAVE_2_] : begin
           if ( slave_req_2 ) begin
              slave_grnt_2 = 1'b1;
              if ( slave_eof_2 ) begin
                 slave_grnt_2 = 1'b0;
                 arb_ns = SLAVE_3;
              end
              else begin
                 arb_ns = SLAVE_2;
              end
           end
           else begin
              arb_ns = SLAVE_3;
           end
        end

        arb_cs[_SLAVE_3_] : begin
           if ( slave_req_3 ) begin
              slave_grnt_3 = 1'b1;
              if ( slave_eof_3 ) begin
                 slave_grnt_3 = 1'b0;
                 arb_ns = SLAVE_4;
              end
              else begin
                 arb_ns = SLAVE_3;
              end
           end
           else begin
              arb_ns = SLAVE_4;
           end
        end

        arb_cs[_SLAVE_4_] : begin
           if ( slave_req_4 ) begin
              slave_grnt_4 = 1'b1;
              if ( slave_eof_4 ) begin
                 slave_grnt_4 = 1'b0;
                 arb_ns = SLAVE_1;
              end
              else begin
                 arb_ns = SLAVE_4;
              end
           end
           else begin
              arb_ns = SLAVE_1;
           end
        end

        default: begin
           arb_ns = 4'hX;
        end
      endcase
   end


endmodule   
\end{lstlisting}



\texttt{`let} directive supports following functions:

\begin{itemize}
\item Numeric operators: addition (\texttt{+}), subtraction (\texttt{-}), multiplication
(\texttt{*}), division (\texttt{/}), modulus (\texttt{\%}), power (\texttt{**}).

\item Logical operators: logical AND (\texttt{\&\&}), logical OR (\texttt{||}), logical
NOT (\texttt{!}).

\item Bit operators: bitwise XOR (\texttt{\textasciicircum{}}), bitwise AND (\texttt{\&}), bitwise OR (\texttt{|}),
shift right (\texttt{>{}>{}}), shift left (\texttt{<{}<{}}).

\item Functions: log 2 (\texttt{LOG2()}), round up (\texttt{CEIL()}), round down (\texttt{FLOOR()}),
round to nearest value (\texttt{ROUND()}), max of two numbers (\texttt{MAX()}), min of
two numbers (\texttt{MIN()}), odd (\texttt{ODD()}), even (\texttt{EVEN()}), absolute value
(\texttt{ABS()}).
\end{itemize}

\begin{lstlisting}[language=verilog,label= ,caption={\texttt{`let} usage examples},captionpos=b,firstnumber=1,numbers=left]
`define x 2

// NOTE `let need "="
`let y = `x ** 10 // now `y is 1024

`let z = LOG2(`y) // `z is 10
`let a = LOG2(`z) // `a is 3.321928
`let c = CEIL(`a) // `c is 4
`let f = FLOOR(`a) // `f is 3
`let r = ROUND(`a) // `r is 3


// we can concatenate macro value with other strings
// using "::" operator
assign cat_`f::k = 1'b0; // expand to ``assign cat_3k = 1'b0;''
assign cat_`fk = 1'b0; // Error!! macro ``fk'' is not defined
\end{lstlisting}

\chapter{User Control}
\label{sec:org17bead1}
mhdl provides control syntax start with keyword \texttt{metahdl}, which
interfaces with and controls the runtime behavior of compiler.
Designers' controls are passed to compiler via variable assignments
embedded in RTL code, this variable settings are also preceded by
keyword \texttt{metahdl}.  Boolean variables inside compiler are set via \texttt{+}
or \texttt{-} preceded by variable name, where \texttt{+} means "enable" and \texttt{-}
means "disable".

There are two special form of control syntax: \texttt{exit} syntax, and
\texttt{message} syntax.  The former is used to command compiler exit when
the statement is encountered.  The latter is used to print messages on
\texttt{stderr}.  They are usually used with preprocessor to guarantee
correct configuration settings.  

Working scope of all variables can be \emph{Modular} or \emph{Effective}.
Modular variables (MVAR) take effect on entire module and are used
when parsing is finished.  Designers can set MVAR anywhere in source
code and get the same effect.  If an MVAR is assigned multiple times,
last assignment wins.  MVAR can have different values in different
files, so file is the minimum granularity of MVAR.

Effective variables (EVAR) take effect from the point the variable is
assigned and are used \emph{during} parsing.  Designers can set different
values for same EVAR in different sections of source code, and make
compiler treat sections differently.  So the minimum granularity of
EVAR is section divided by EVAR assignments.

\section{Variable List}
\label{sec:orge6b7b18}
Following is the complete list of all compiler variables can be assigned
by user control syntax, variable type (boolean or string) and variable
scope (MVAR or EVAR) are listed with variable name.

\begin{longtable}{|l|c|c|l|p{.45\textwidth}|}
\caption{Compiler Control variables  \label{tab:com_ctrl}}
\\[0pt]
\hline
\rowcolor{ThemeColor}\tableheader{Name} & \tableheader{Scope} & \tableheader{Type} & \tableheader{Default} & \tableheader{Description}\\[0pt]
\hline
\endfirsthead
\multicolumn{5}{l}{Continued from previous page} \\[0pt]
\hline

\rowcolor{ThemeColor}\tableheader{Name} & \tableheader{Scope} & \tableheader{Type} & \tableheader{Default} & \tableheader{Description }\\[0pt]

\hline
\endhead
\hline\multicolumn{5}{r}{Continued on next page} \\
\endfoot
\endlastfoot
\hline
modname & MVAR & string & Base file name & Set the generated module name. Often used with preprocessor to distinguish modules with different configurations.\\[0pt]
outfile & MVAR & string & Base file name & Set the generated Synthesizable file base name. Often used with preprocessor to distinguish module definition files with different configurations.\\[0pt]
portchk & MVAR & boolean & false & Enable/Disable port validation for module.\\[0pt]
hierachydepth & MVAR & postitive int & 300 & Maximum level of module instantiation.\\[0pt]
clock & EVAR & string & clock & Default clock name used for \texttt{ff-endff} and \texttt{fsm-endfsm}\\[0pt]
reset & EVAR & string & reset\_n & Default reset name used for \texttt{ff-endff} and \texttt{fsm-endfsm}\\[0pt]
relexedfsm & EVAR & boolean & true & Set severity of connectivity/reachability error checked in FSM. If it is true, relaxed FSM programming mode is enabled, all dead states or unreachable states are acceptable, compiler only reports warning when such states are encountered, and continues processing. if it is false, FSM programming is in strict mode, any dead state or unreachable state is considered to be fatal error, compiler will report error and stop processing if such state is checked.\\[0pt]
exitonwarning & EVAR & boolean & false & Set severity of \emph{normal parsing warning}, such as width mismatch. If it is true, compiler exits on any warning.\\[0pt]
exitonlintwarning & EVAR & boolean & false & Warning from port validation, multiple driver checking are categorized as lint warning. If this variable is set to true, compiler will exit on any lint warning.\\[0pt]
\hline
\end{longtable}


\section{Example}
\label{sec:org7f63deb}
demonstrates user control syntax with code configuration.

\begin{itemize}
\item Line 1 enables port validation in this module.
\item Line 3 checks value of macro \textasciitilde{}`WIDTH`, forces compilation exit upon
illegal values.
\item Line 16 and 17 alter module name and output file name according to
target device.
\item Any warning between 19 and 29 makes compiler exit.  To be more
specific, width mismatch between \texttt{b\_ff} and \texttt{b} is considered to be
fatal error.
\item Different clock and reset names are used for different code sections.
\end{itemize}

\begin{lstlisting}[language=verilog,label= ,caption= ,captionpos=b,firstnumber=1,numbers=left]
metahdl + portchk;

`if WIDTH > 64
metahdl ``width can not exceed 64!'';
metahdl exit;
`endif

assign data[`WIDTH-1:0] = `WIDTH'd0;

`ifdef FPGA
 `define target fpga
`else
 `define target asic
`endif

metahdl modname = top_`target;
metahdl outfile = top_`target;

metahdl + exitonwarning;

metahdl clock = clk_125M;
metahdl reset = pclk_rst_n;

ff;
a_ff, a, 1'b0;
b_ff[1:0], b, 1'b0;
endff

metahdl - exitonwarning;

metahdl clock = clk_250M;
metahdl reset = dclk_rst_n;

ff;
c_ff, c, 1'b0;
d_ff, d, 1'b0;
endff

ff;
e_ff, e, 1'b0;
g_ff, g, 1'b0;
endff
\end{lstlisting}


\chapter{Compiler Usage}
\label{sec:org32ec4ba}
\section{Command Line Options}
\label{sec:orgcd94bb9}
\texttt{mhdlc} command line is listed below:
\begin{verbatim}
% mhdlc <options> mhdl_file [mhdl_file mhdl_file ...]
\end{verbatim}

Command line captions are case sensitive, which means \texttt{-p} and \texttt{-P}
are different.  Major options are listed below:

\begin{enumerate}
\item \texttt{-I} Specify single search path for \texttt{mhdlc} to look for \texttt{`include}
files.

\item \texttt{-o} Specify output base directory for generated files.  mhdl source
directory structure is mirrored under it.

\item \texttt{-D} Define macro from command line. e.g., \texttt{-DCMD\_EN} defines a
macro named \texttt{CMD\_EN}, \texttt{-DSLV\_NUM=4} defines a macro \texttt{SLV\_NUM} with
\texttt{4} as its value.

\item \texttt{-mb} specify mhdl source base directory.  All mhdl files are
searched recursively from this directory.  \texttt{-mb} can only be
specified once, multiple values are not allowed.

\item \texttt{-ib} specify IP base directories, which contain existing verilog
designs.  Multiple \texttt{-ib} can be spcified at command line.
\end{enumerate}

All other text in command line and does not start with "-" are
considered to be file names to be processed.

\section{For VPerl Designers}
\label{sec:orgd4e6ee2}
\begin{enumerate}
\item \texttt{\&Depend} is no longer needed since automatically resolves
dependency.

\item \texttt{vpmake -depend} is not needed anymore, just give top level file and
search path.

\item All \texttt{\&Force} should be converted to standard declarations, including
2D array.

\item \texttt{\&ConnRule} and \texttt{\&Connect} should be converted according to port
connect syntax.

\item \texttt{\&Instance} should be converted to instantiation syntax.

\item \texttt{c-sky vperl\_off} and \texttt{c-sky vperl\_on} should be converted to
\texttt{rawcode} and \texttt{endrawcode}.

\item File extension is \texttt{.mhdl}, not \texttt{.vp} .
\end{enumerate}


\chapter{Formal Syntax}
\label{sec:orgbcbe461}
\input{formal_syntax.tex}
\end{document}
